%!TEX root = thesis.tex

\chapter{Einleitung}

Die Einleitung führt zum eigentlichen Thema dieser Arbeit hin. Dabei wird ein großer Bogen gespannt, in dem die Relevanz und der Kontext der untersuchten Thematik deutlich wird. Grundlegende Begriffe aus dem Titel und der Kurzfassung sollten aufgegriffen und definiert werden. Unterstützend können Zitate herangezogen werden, die der Arbeit einen Rahmen geben.
\section{Hinweise zur Erstellung einer Arbeit am Institut für Physik}

\textbf{Beachten Sie bei der Anfertigung Ihrer Arbeit bitte folgende Punkte:}\vspace{0.5cm}
\begin{itemize}
\item \textbf{Strategie}\begin{itemize}
	\item man beginnt mit den Ergebnissen, d. h. meist mit Graphen/Bildern, die die Ergebnisse zeigen
	\item dann Beschreibung der Ergebnisse, was kann man schlussfolgern? (hier fallen evtl. Messungen auf, die wiederholt werden müssten, oder Dinge die noch fehlen)
	\item dann der Rest (Grundlagen, Material und Methoden)
	\item erst ganz am Ende: Diskussion, Abstract/Zusammenfassung (d. und eng.), Einleitung (Motivation)
\end{itemize}
\item \textbf{Inhalt}\begin{itemize}

\item Aufbau der Arbeit/Inhaltsverzeichnis mit dem Betreuer besprechen
\item in die Grundlagen (physikalische/biochemische...) gehört dann das, was man zum Erklären der Ergebnisse benötigt, nicht mehr
\item die Einleitung (Motivation) beschreibt kurz (mit rotem Faden...\textbf{Geschichte erzählen})
\begin{itemize}
\item ganz allgemeiner Stand der Wissenschaft (eins/zwei Sätze),
\item	die Problemstellung bei diesem Stand,
\item	die Methode, mit der man die Problemstellung bearbeitet,
\item	was genau untersucht wird, 
\item	und was das Ziel der vorliegenden Arbeit ist (ohne Ergebnisse)
\end{itemize} 
\item Abstract/Zusammenfassung enthält kurze Motivation, Methoden und Ergebnisse
\end{itemize}

\item \textbf{Stil und Layout:}\begin{itemize}
	\item Benutzen Sie das DIN-A4-Format.
	\item Schreiben Sie im Blocksatz.
	\item Achten Sie auf einen ausreichenden Seitenrand .
	\item Benutzen Sie keine serifenlose Schriftart. Mögliche Serif-Schriftarten sind zum Beispiel Times, Computer Modern, Baskerville oder Garamond.
	\item Nutzen Sie den normalen (einfachen) Zeilenabstand. Sollte der Text dadurch sehr unübersichtlich erscheinen, erhöhen Sie den Zeilenabstand auf 1,1- bis 1,2-fach.
	\item Schreiben Sie präzise. Umschreiben Sie keine Sachverhalte, die Sie auch konkret benennen können.
	\item Drücken Sie sich möglichst im Passiv aus (nicht "`man"'). Vermeiden Sie Sätze im Imperativ, Umgangssprache und unnötige Füllwörter.
	\item "`Laborslang"' und englische Begriffe vermeiden, z. B. nicht Set-up sondern Aufbau oder anstatt "`mit Zaubercocktail"' "`mit ROXS"'
	\item Erklären Sie Abkürzungen bei ihrer ersten Verwendung.
	\item Lateinische Begriffe immer kursiv, z. B. \textit{in vivo}.
	\end{itemize}

\item \textbf{Zahlen und Einheiten:}\begin{itemize}
	\item Physikalische \textbf{Größen} sind in Formeln und im Fließtext \textbf{kursiv}, z. B. Temperatur $T$ und Zeit $t$. 
	\item \textbf{Indices} von physikalischen Größen, die keine Variablen sind, sind \textbf{nicht kursiv}, z. B. $T_{\rm{1}}$, Zeit $t_{\rm{off}}$ (hingegen $T_x$ oder $T_i$). 
	\item Einheitenbehaftete Zahlen werden immer mit ihrer Einheit geschrieben.
	\item Zwischen Zahl und Einheit steht ein geschütztes Leerzeichen. Ein geschütztes Leerzeichen verhindert, dass Zahl und Einheit in unterschiedlichen Zeilen stehen und fügt immer den gleichen Abstand zwischen Zahl und Einheit ein.
	\item \textbf{Einheiten} werden \textbf{nicht kursiv} geschrieben. 
	\item \textbf{Einheiten} werden bei der Beschriftung von Graphen oder in Tabellen \textbf{nicht in eckige Klammern} gesetzt. Bsp: $c$ in nM oder $c /$ mM!
	\end{itemize}
\item \textbf{Abbildungen und Tabellen:}\begin{itemize}
	\item Abbildungen und Tabellen werden fortlaufend nummeriert und immer beschriftet. 
	\item Die Beschriftung von Abbildungen erfolgt unterhalb der Abbildung. Die Beschriftung von Tabellen erfolgt oberhalb der Tabelle.
	\item Auf jede Abbildung und Tabelle wird mindestens einmal im Text verwiesen.
	\item Die Beschriftungen der Graphen sollten so ausführlich sein, dass die Ergebnisse einer Arbeit auch nur mit den Bildern verständlich sind!
	\item Von Anfang an auf die Größe von Graphen und deren Beschriftung achten (am besten ist, den Graph in der Größe zu erstellen, in der er ins Skript kommt). \textbf{Schriftgröße} im Graphen an \textbf{Text anpassen}, nicht zu klein oder zu groß!
	\end{itemize}
\item \textbf{Zitate und Quellenangaben:}\begin{itemize}
	\item Wörtlich übernommene Texte werden durch Anführungsstriche kenntlich gemacht und vom restlichen Text abgesetzt.
	\item Formulieren Sie nach Möglichkeit alles in eigenen Worten.
	\item Literaturstellen werden fortlaufend nummeriert und und in eckigen Klammern hinter die entsprechende Textstelle gesetzt (am Satzende vor dem Punkt). 
	\item Die Quellen stehen am Ende der Arbeit in einem Literaturverzeichnis in der Reihenfolge der vergebenen Nummern.
	\end{itemize}
\end{itemize}

\section{Verwandte Arbeiten}

Eine wichtiger Abschnitt der Einleitung stellt einen Überblick über verwandte Arbeiten dar. Was wurde bereits in der Literatur untersucht und ist \emph{nicht} Thema dieser Arbeit?

\section{Aufbau der Arbeit}

Neben dieser Einleitung und der Zusammenfassung am Ende gliedert sich diese Arbeit in die folgenden drei Kapitel.
\begin{description}
  \item[\ref{chapter-basics}] beschreibt die für diese Arbeit benötigten Grundlagen. In diesem Kapitel werden \ldots, \ldots und \ldots eingeführt, da diese für die folgenden Kapitel dringend benötigt werden.
  \item[\ref{chapter-konzept}] stellt das eigentliche Konzept vor. Dabei handelt es sich um ein Konzept zur Verbesserung der Welt. Das Kapitel gliedert sich daher in einen globalen und einen lokalen Ansatz, wie die Welt zum Besseren beeinflusst werden kann.
  \item[\ref{chapter-evaluation}] beinhaltet eine Evaluation des Konzeptes aus dem vorherigen Kapitel. Anhand von Simulationen wird in diesem Kapitel untersucht, wie die Welt durch konkrete Maßnahmen deutlich verbessert werden kann.
\end{description}

