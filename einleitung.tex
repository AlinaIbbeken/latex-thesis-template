%!TEX root = thesis.tex

\chapter{Einleitung}

Die Einleitung führt zum eigentlichen Thema dieser Arbeit hin (Motivation). Dabei wird ein großer Bogen gespannt, in dem die Relevanz und der Kontext der untersuchten Thematik deutlich wird. Grundlegende Begriffe aus dem Titel und der Kurzfassung sollten aufgegriffen und definiert werden.

Allgemein sollte versucht werden einen \textbf{roten Faden} zu legen, der sich durch die gesamte Arbeit zieht. Ein sinnvoller Aufbau, der dem \textbf{Erzählen einer Geschichte} gleichkommen sollte, erleichtert das Lesen und das Nachvollziehen der Thematik und Ergebnisse.

Die Gliederung der Arbeit und die Benennung und Strukturierung einzelner Kapitel kann und sollte in Absprache mit dem Betreuer angepasst werden. So sollte sich beispielsweise der Aufbau einer Arbeit mit biochemischem Schwerpunkt von dem einer Arbeit mit Schwerpunkt auf technischer Konstruktion oder Programmierung unterscheiden.

Es bietet sich an, die Einleitung in Unterkapitel aufzuteilen. Diese sollten die folgenden Punkte knapp erörtern:

\begin{description}
	\item \textbf{Stand der Technik/Wissenschaft}, wie war der Wissensstand vor dieser Arbeit?
	\item \textbf{Problematik}, wo setzt diese Arbeit an, welches Problem soll sie lösen?
	\item \textbf{Methode}, wie soll das Wissen erweitert werden, wie soll das Problem gelöst werden?
\end{description}

Zum Beispiel wie im Folgenden.

\section{Stand der Wissenschaft}

Ein wichtiger Abschnitt der Einleitung stellt einen Überblick über verwandte Arbeiten dar. Was wurde bereits in der Literatur untersucht und ist \emph{nicht} Thema dieser Arbeit?

\section{Ziele der Arbeit}

Welches Problem, oder welche Probleme, sollen im Rahmen dieser Arbeit gelöst werden? Was soll wiederholt und überprüft werden? Welches Ergebnis soll im Besten Fall am Ende dieser Arbeit stehen?

\section{Vorgehensweise}

Wie soll vorgegangen werden um das zuvor definierte Ziel zu erreichen? Welche Methodiken sollen angewendet werden? Was muss zunächst analysiert und überprüft werden, um zielorientiert darauf aufbauen zu können? Welche Experimente und Messungen sollen durchgeführt werden?

Folgende kurze Darstellung der Inhalte einzelner Kapitel sollte am Ende dieses Abschnitts (Vorgehensweise) eingearbeitet werden.

Neben der Einleitung und der Zusammenfassung am Ende gliedert sich diese Arbeit in die folgenden Kapitel.
\begin{description}
  \item[\ref{chapter:basics}] beschreibt die für diese Arbeit benötigten Grundlagen. In diesem Kapitel werden \ldots, \ldots und \ldots eingeführt, da diese für die folgenden Kapitel dringend benötigt werden.
  \item[\ref{chapter:material-methoden}] listet die Materialien auf, Methoden...
  \item[\ref{chapter:ergebnisse}] stellt das eigentliche Konzept vor. Dabei handelt es sich um ein Konzept zur Verbesserung der Welt. Das Kapitel gliedert sich daher in einen globalen und einen lokalen Ansatz, wie die Welt zum Besseren beeinflusst werden kann.
  \item[\ref{chapter:diskussion}] beinhaltet eine Evaluation des Konzeptes aus dem vorherigen Kapitel. Anhand von Simulationen wird in diesem Kapitel untersucht, wie die Welt durch konkrete Maßnahmen deutlich verbessert werden kann.
\end{description}

