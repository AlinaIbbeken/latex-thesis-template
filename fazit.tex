%!TEX root = thesis.tex

\chapter{Zusammenfassung und Ausblick}
\label{chapter-fazit}

Die Zusammenfassung greift die in der Einleitung angerissenen Bereiche wieder auf und erläutert, zu welchen Ergebnissen diese Arbeit kommt. Dabei wird insbesondere auf die neuen Erkenntnisse und den Nutzen der Arbeit eingegangen.

Im anschließenden Ausblick werden mögliche nächste Schritte aufgezählt, um die Forschung an diesem Thema weiter voranzubringen. Hier darf man sich nicht scheuen, klar zu benennen, was im Rahmen dieser Arbeit nicht bearbeitet werden konnte und wo noch weitere Arbeit notwendig ist.

\section{Credits}

Diese \LaTeX\ -Vorlage wurde von Malte Schmitz in Anlehnung an die Vorgaben der Sektion MINT der Universität zu Lübeck erstellt, und ist in ihrer unveränderten Version in seinem GitHub-Repository verfügbar: \\
\url{https://github.com/malteschmitz/latex-thesis} \\

Ich (Janosch Kappel) habe die Vorlage lediglich so verändert, und mit ergänzenden Hinweisen versehen, dass sie den Ansprüchen an eine Abschlussarbeit am Institut für Physik der Universität zu Lübeck genügt. Dazu habe ich Hinweistexte aus einer älteren Vorlage (erstellt von Jan Pavlita), sowie Auszüge aus einem Leitfaden der Universität Bremen \cite{uni-bremen} verwendet. \\

Die Vorlage kann unter folgendem Link auf GitHub gefunden werden:\\
\url{https://github.com/JaKaPhysics/latex-thesis-template}

\section{Hilfreiches Material}

An dieser Stelle sei noch einmal auf den hilfreichen und umfassenden Leitfaden zum Verfassen einer Abschlussarbeit der Universität Bremen hingewiesen \cite{uni-bremen}. \\

Für einen angenehmen und umfassenden Einstieg in \LaTeX\ empfehle ich die exzellenten Ausführungen und Einführungs-Videos von Malte Schmitz: \\
\url{https://www.mlte.de/latex/latex-talk/} \\
\url{https://www.mlte.de/videos/latex/} \\

Es lohnt sich auch sehr einen Blick in die Hinweistexte der von Till Tantau erstellten Vorlage für Abschlussarbeiten zu werfen: \\
\url{https://moodle.uni-luebeck.de/course/view.php?id=4439}
