%!TEX root = thesis.tex

\chapter{Material und Methoden}
\label{chapter:material-methoden}

In diesem Abschnitt sollte konkret und nachvollziehbar beschrieben werden, wie die Untersuchungen durchgeführt wurden. So sollte zusätzlich zu einer Auflistung der verwendeten Materialien (Chemikalien und Geräte inkl. eventueller Messungenauigkeiten und Firmennamen) beschrieben werden, wie beispielsweise Proben hergestellt wurden (z.B. Konzentrationen, Verdünnungsreihen, etc.) und mit welchen Parametern Geräte verwendet wurden (z.B. erforderliche Zeit im Ultraschallbad, Porengröße der Extrudermembran). \textbf{Die durchgeführten  Arbeiten müssen hierbei durch andere reproduzierbar sein.}

Die folgenden Abschnitte dieses Kapitels enthalten Beispiele für die diversen Inhaltselemente einer Arbeit.\todo{Die Abschnitte dieses Kapitels sollten natürlich nicht so in die Arbeit übernommen werden. Für die finale Version können diese Todo-Notes dann komplett aufgeblendet werden}

\todo[inline]{Notizen an einen selbst oder den Betreuer der Arbeit sind während der Arbeit sehr nützlich.}

\section{Quellen}

Quellen sind wichtig für gutes wissenschaftliches Arbeiten. Eine Quelle kann dabei zum Beispiel
\begin{compactitem}
  \item ein Beitrag in einer Zeitschrift \cite{MopOverview},
  \item ein Beitrag in einem Sammlungsband \cite{moore},
  \item ein Buch \cite{scala},
  \item ein Beitrag im Berichtsband einer Konferenz \cite{rltl},
  \item ein technischer Bericht \cite{bitkom},
  \item eine Dissertation \cite{Leucker02},
  \item eine Abschlussarbeit \cite{RltlConv},
  \item ein (noch) nicht veröffentlichter Artikel \cite{ptLTL} oder
  \item ein Artikel auf einer Website \cite{codecommit} sein.
\end{compactitem}

Dabei ist zu beachten, dass nicht veröffentlichte Artikel und insbesondere Webseiten nur in Ausnahmefällen gute Quellen sind, da diese nicht durch Fachleute begutachtet wurden.

In den meisten Fällen können Quellenangaben im Bib\TeX-Format direkt in verschiedenen Suchmaschinen\footnote{zum Beispiel Google Scholar (\url{https://scholar.google.com/}) oder PubMed (\url{https://www.ncbi.nlm.nih.gov/pubmed/})} für wissenschaftliche Texte entnommen werden.

\section{Tabellen}

In \vref{tbl:prozessoren} sehen wir ein Beispiel für eine Tabelle. Im Gegensatz zu Abbildungen und Diagrammen ist die Beschriftung einer Tabelle immer oberhalb positioniert.

\begin{table}

  \centering
  \begin{tabular}{llr}
    \headerrow Jahr & Prozessor & MHz \\
    1975 & 6502 (C64) 	& 1 \\
    1985 & 80386 			& 16 \\
    2005 & Pentium 4 	& 2\,800 \\
    2030 & Phoenix 3 	& 7\,320\,000 \\
    \hiderowcolors
    2050 & \ldots \\
    2070 & \ldots
  \end{tabular}
  \caption[Rechengeschwindigkeit von Computern]{Rechengeschwindigkeit von Computern. Inhaltlich vollkommen egal, ist dies doch ein sehr schönes Beispiel für eine Tabelle.}
  \label{tbl:prozessoren}
\end{table}

\section{Abbildungen und Diagramme}

In \vref{fig:flower} sehen wir ein Beispiel für eine Abbildung, die aus einer externen Grafik geladen wurde. In \vref{fig:buechi} sehen wir ein Beispiel für eine Abbildung, die in mit Hilfe von TiKz in \LaTeX\ generiert wurde.

%\item \textbf{Abbildungen und Tabellen:}\begin{itemize}
%	\item Abbildungen und Tabellen werden fortlaufend nummeriert und immer beschriftet. 
%	\item Die Beschriftung von Abbildungen erfolgt unterhalb der Abbildung. Die Beschriftung von Tabellen erfolgt oberhalb der Tabelle.
%	\item Auf jede Abbildung und Tabelle wird mindestens einmal im Text verwiesen.
%	\item Die Beschriftungen der Graphen sollten so ausf�hrlich sein, dass die Ergebnisse einer Arbeit auch nur mit den Bildern verst�ndlich sind!
%	\item Von Anfang an auf die Gr��e von Graphen und deren Beschriftung achten (am besten ist, den Graph in der Gr��e zu erstellen, in der er ins Skript kommt). \textbf{Schriftgr��e} im Graphen an \textbf{Text anpassen}, nicht zu klein oder zu gro�!
%	\end{itemize}
%\item \textbf{Zitate und Quellenangaben:}\begin{itemize}
%	\item W�rtlich �bernommene Texte werden durch Anf�hrungsstriche kenntlich gemacht und vom restlichen Text abgesetzt.
%	\item Formulieren Sie nach M�glichkeit alles in eigenen Worten.
%	\item Literaturstellen werden fortlaufend nummeriert und und in eckigen Klammern hinter die entsprechende Textstelle gesetzt (am Satzende vor dem Punkt). 
%	\item Die Quellen stehen am Ende der Arbeit in einem Literaturverzeichnis in der Reihenfolge der vergebenen Nummern.
%	\end{itemize}
%\end{itemize}

\begin{figure}
  \centering
  \pgfimage[width=.5\textwidth]{flower}
  \caption[Kurzfassung der Beschreibung für das Abbildungsverzeichnis]{Lange Version der Beschreibung, die direkt unter der Abbildung gesetzt wird. Es ist wichtig, für jede Abbildung eine umfangreiche Beschreibung anzugeben, da der Leser beim ersten Durchblättern der Arbeit vor allem an den Abbildungen hängen bleibt. Am Besten sollte so die Abbildung mit der dazugehörigen Bildunterschrift ohne weiteren Text verständlich sein und für sich alleine stehen können.}
  \label{fig:flower}
\end{figure}

\begin{figure}
  \centering
  \begin{tikzpicture}[
      node distance=15ex,
      auto,
      on grid,
      shorten >=1pt
    ]
    \node [state, initial] (q0) {$q_0$};
    \node [state, accepting, right=of q0] (q1) {$q_1$};
    \path[->]
      (q0) edge node {$a$} (q1);
  \end{tikzpicture}
  \caption[Graph des Büchi-Automaten $\hat A$.]{Graph des Büchi-Automaten $\hat A$. Der Zustand $q_1$ hat dabei keine ausgehende Kante. Der Zustand ist trotzdem akzeptierend, da beide enthaltenen Zustände von $\acute A$ akzeptierend sind. Die naive Anwendung des Leerheitstests auf alternierenden Büchi-Automaten liefert in diesem Fall also zu viele akzeptierende Zustände.}
  \label{fig:buechi}
\end{figure}

\section{Quelltext}

Quelltext sollte in Abschlussarbeiten nur äußerst sparsam eingesetzt werden. Wichtig ist insbesondere, dass Quelltextauszüge sorgsam ausgewählt und gut erklärt werden.

\begin{lstlisting}[language=Java,gobble=2]
  public class Main {
    // Hello Word in Java
    public static void main(String[] args) {
      System.out.println("Hello World");
    }
  }
\end{lstlisting}

\subsection{Quelltext mit automatischer Nummerierung}

Manchmal möchte man Quelltext eher als Abbildung und nicht als Fließtext behandeln. In diesem Fall soll der Quelltext eine Bildunterschrift und eine automatische Nummerierung erhalten. Die automatische Nummerierung kann dann natürlich auch in Referenzen verwendet werden: In \vref{lst:java} haben wir Java-Quelltext.

\begin{lstlisting}[language=Java,gobble=2,caption={Ich bin die Bildunterschrift des Quelltextes},label=lst:java]
  public class AnotherClass {
    private int number = 0;
    public void add() {
      this.number++;
    }
  }
\end{lstlisting}

Wenn man Quelltext mit Bildunterschrift setzt, muss man darauf achten, dass der Quelltext nach wie vor nicht als Fließumgebung behandelt wird. Entsprechend kann es passieren, dass der Quelltext über zwei Seiten hinweg gesetzt wird. Während das normalerweise nicht stört, kann dieser Umstand in Zusammenhang mit einer Bildunterschrift den Leser irritieren.

\section{Formeln mit \pdfepsilon}

Das $\epsilon$ in der Überschrift dieses Abschnitts ist ein Beispiel für ein mathematisches Symbol, dass in den PDF-Lesezeichen als reiner Text gesetzt wird. Siehe \texttt{macros.tex}. In dieser Datei wird auch $n \in \N$ definiert.

Wir wissen aus der Analysis, dass
\begin{align}
  f(x) &= x^2 + px + q
\end{align}
Nullstellen bei
\begin{align}
  x_1 &= -\frac p2 + \sqrt{\frac{p^2}4 - q} \text{ und}\\
  x_2 &= -\frac p2 - \sqrt{\frac{p^2}4 - q}
\end{align}
hat.

%\item \textbf{Zahlen und Einheiten:}\begin{itemize}
%	\item Physikalische \textbf{Gr��en} sind in Formeln und im Flie�text \textbf{kursiv}, z. B. Temperatur $T$ und Zeit $t$. 
%	\item \textbf{Indices} von physikalischen Gr��en, die keine Variablen sind, sind \textbf{nicht kursiv}, z. B. $T_{\rm{1}}$, Zeit $t_{\rm{off}}$ (hingegen $T_x$ oder $T_i$). 
%	\item Einheitenbehaftete Zahlen werden immer mit ihrer Einheit geschrieben.
%	\item Zwischen Zahl und Einheit steht ein gesch�tztes Leerzeichen. Ein gesch�tztes Leerzeichen verhindert, dass Zahl und Einheit in unterschiedlichen Zeilen stehen und f�gt immer den gleichen Abstand zwischen Zahl und Einheit ein.
%	\item \textbf{Einheiten} werden \textbf{nicht kursiv} geschrieben. 
%	\item \textbf{Einheiten} werden bei der Beschriftung von Graphen oder in Tabellen \textbf{nicht in eckige Klammern} gesetzt. Bsp: $c$ in nM oder $c /$ mM!
%	\end{itemize}

\section{Hinweise zu Form und Stil}

%\begin{itemize}
%	\item Schreiben Sie pr�zise. Umschreiben Sie keine Sachverhalte, die Sie auch konkret benennen k�nnen.
%	\item Dr�cken Sie sich m�glichst im Passiv aus (nicht "`man"'). Vermeiden Sie S�tze im Imperativ, Umgangssprache und unn�tige F�llw�rter.
%	\item "`Laborslang"' und englische Begriffe vermeiden, z. B. nicht Set-up sondern Aufbau oder anstatt "`mit Zaubercocktail"' "`mit ROXS"'
%	\item Erkl�ren Sie Abk�rzungen bei ihrer ersten Verwendung.
%	\item Lateinische Begriffe immer kursiv, z. B. \textit{in vivo}.
%	\end{itemize}
