%!TEX root = thesis.tex

\chapter{Ergebnisse}
\label{chapter:ergebnisse}

In diesem Kapitel werden die Ergebnisse dieser Arbeit in geordneter Reihenfolge dargestellt. Es sollte dabei zunächst nur beschrieben und nicht interpretiert werden. Die Interpretation der Ergebnisse und kritische Auseinandersetzung mit Ergebnissen und Methoden sind Bestandteil der Diskussion.
Je nach Schwerpunkt der Arbeit können die Kapitel zu Ergebnissen und Diskussion auch anders gestaltet werden, beispielsweise wenn sich die Arbeit auf Entwurf und Realisierung eines technischen Aufbaus konzentriert. 

\section{Messergebnisse}

Der folgende Abschnitt ist einem Leitfaden der Universität Bremen \cite{uni-bremen} entnommen:
Der Text des Ergebnisteils sollte im Ablauf so dargestellt werden, wie die Fragen und Hypothesen eingangs gestellt wurden. Größere Listen und Tabellen mit Originaldaten, die im Hauptteil mehrere Seiten einnehmen würden, werden in den Anhang platziert. Sind Ergebnisse aus Ihrer Arbeit von untergeordneter Bedeutung und tragen nicht zum Verständnis des wirklich Wichtigen bei, sollten diese ebenfalls nur im Anhang gezeigt werden. Vorsicht, auch auffällig negative Ergebnisse müssen erwähnt und dürfen nicht verschwiegen werden.
Den Ergebnissen werden Grafiken und/oder Tabellen beigeordnet, die den Text veranschaulichen sollen. Dabei ist zu beachten, dass auf die Abbildungen und Tabellen auch im Text verwiesen wird.

\section{Implementierungen}

Wenn Implementierungen umfangreich beschrieben werden, ist darauf zu achten, den richtigen Mittelweg zwischen einer zu detaillierten und zu oberflächlichen Beschreibung zu finden. Eine Beschreibung aller Details der Implementierung ist in der Regel zu detailliert, da die primäre Zielgruppe einer Abschlussarbeit sich nicht im Detail in den geschriebenen Quelltext einarbeiten will. Die Beschreibung sollte aber durchaus alle wesentlichen Konzepte der Implementierung enthalten. In einer solchen Beschreibung sind auch einige unterstützende Diagramme hilfreich.

